\documentclass[14pt, a4paper]{extarticle} % или article, если не нужен увеличенный базовый шрифт
\usepackage[utf8]{inputenc}
\usepackage[T2A]{fontenc}
\usepackage[russian]{babel}
\usepackage{array}
\usepackage{graphicx}
\usepackage{ulem}
\usepackage{geometry}
\geometry{top=2cm, bottom=2cm, left=2cm, right=2cm}
\usepackage{tabularx}

% Для подчеркивания текста, аналогичного .underline в Word
\newcommand{\und}[1]{\uline{#1}}

\begin{document}

% Титульный лист
\begin{titlepage}
    \centering
    \noindent
    \begin{tabular}{@{}p{0.25\textwidth} p{0.7\textwidth}@{}}
        \raisebox{-\totalheight}{\includegraphics[width=0.8in, height=0.9in]{imgs/BMSTU.jpg}} &
        \textbf{Министерство науки и высшего образования Российской Федерации} \\
        & \\
        & \textbf{Федеральное государственное бюджетное образовательное} \\
        & \textbf{учреждение} \\
        & \\
        & \textbf{высшего образования} \\
        & \\
        & \textbf{«Московский государственный технический университет} \\
        & \\
        & \textbf{имени Н.Э. Баумана} \\
        & \\
        & \textbf{(национальный исследовательский университет)»} \\
        & \\
        & \textbf{(МГТУ им. Н.Э. Баумана)} \\
    \end{tabular}

    \vspace{1cm}
    ФАКУЛЬТЕТ \und{«СПЕЦИАЛЬНОЕ МАШИНОСТРОЕНИЕ»}

    \vspace{0.5cm}
    КАФЕДРА \und{«РАКЕТНЫЕ И ИМПУЛЬСНЫЕ СИСТЕМЫ» (СМ-6)}

    \vspace{2cm}
    \textbf{РАСЧЕТНО-ПОЯСНИТЕЛЬНАЯ ЗАПИСКА}

    \vspace{1cm}
    \textbf{\textit{К КУРСОВОЙ РАБОТЕ}}

    \vspace{1cm}
    \textbf{\textit{НА ТЕМУ:}}

    \vspace{0.5cm}
    \textbf{\textit{\und{«Определение рациональных параметров метательных устройств на сжатом газе»}}}

    \vspace{1.5cm}
    \textbf{Вариант 21}

    \vspace{2cm}
    \begin{flushleft}
        Студент \und{СМ6-52} \hfill \underline{\hspace{5cm}} \hfill \und{Д. М. Брюшков} \\
        \hspace{4.8cm} (Группа) \hspace{2.3cm} (Подпись, дата) \hspace{1.8cm} (И.О.Фамилия) \\[1cm]
        Руководитель курсовой работы \hfill \underline{\hspace{5cm}} \hfill \und{В.А. Федулов} \\
        \hspace{7.5cm} (Подпись, дата) \hspace{1.8cm} (И.О.Фамилия)
    \end{flushleft}

    \vfill
    \centering
    \textit{2024 г.}

    \newpage % Начало второй страницы (Задание)

    \centering
    \textbf{Министерство науки и высшего образования Российской Федерации}

    \textbf{Федеральное государственное бюджетное образовательное учреждение}

    \textbf{высшего образования}

    \textbf{«Московский государственный технический университет имени Н.Э. Баумана}

    \textbf{(национальный исследовательский университет)»}

    \textbf{(МГТУ им. Н.Э. Баумана)}

    \vspace{1cm}
    УТВЕРЖДАЮ

    \vspace{0.5cm}
    Заведующий кафедрой \und{СМ6} \\
    \hspace{1.5cm} (Индекс) \\[0.5cm]
    \underline{\hspace{5cm}} \und{В.М.Кашин} \\
    \hspace{0.5cm} (И.О.Фамилия) \\[0.5cm]
    « \underline{\hspace{1cm}} » \underline{\hspace{3cm}} 20 \underline{\hspace{1cm}} г.

    \vspace{2cm}
    \textbf{ЗАДАНИЕ}

    \textbf{на выполнение курсовой работы}

    \vspace{1cm}
    \begin{flushleft}
        по дисциплине \und{Газовая динамика} \hfill \underline{\hspace{7cm}} \\[0.5cm]
        Студент группы \und{СМ6-52} \\[0.5cm]
        \underline{\hspace{12cm}} \und{Брюшков Дмитрий Максимович} \\
        \hspace{5cm} (Фамилия, имя, отчество) \\[0.5cm]
        Тема курсовой работы \und{Определение рациональных параметров метательных устройств} \\
        \und{на сжатом газе} \\[0.5cm]
        Направленность КР (учебная, исследовательская, практическая, производственная, др.) \\
        \und{Учебная} \hfill \underline{\hspace{7cm}} \\[0.5cm]
        Источник тематики (кафедра, предприятие, НИР) \und{Кафедра} \hfill \underline{\hspace{5cm}} \\[0.5cm]
        График выполнения работы: 25\% к \und{4} нед., 50\% к \und{8} нед., 75\% к \und{12} нед., 100\% к \und{14} нед. \\[1cm]
        \textbf{\textit{Задание}} \und{Найдите минимальное начальное давление газа, при котором удается обеспечить скорость метаемо-го тела vpm тела массой m из ствола калибром d. Общая длина тру-бы не должна превышать n калибров. Наибольшая длина камеры не более 1/2 общей длины трубы. Начальная плотность газа не бо-лее $\rho_0$ max. Газ считать холодным (T0 = 300 K), метод лагранжевых координат.} \\[1cm]
        \textbf{\textit{Оформление курсовой работы:}} \\[0.5cm]
        Расчетно-пояснительная записка на \underline{\hspace{2cm}} листах формата А4. \\[0.5cm]
        \underline{\hspace{15cm}} \\
        \underline{\hspace{15cm}} \\
        \underline{\hspace{15cm}} \\
        \underline{\hspace{15cm}} \\[1.5cm]
        Дата выдачи задания «\und{13}» \und{сентября} \und{2024} г. \\[2cm]
        \textbf{Руководитель курсовой работы} \hfill \underline{\hspace{5cm}} \hfill \und{В.А. Федулов} \\
        \hspace{8cm} (Подпись, дата) \hspace{2cm} (И.О.Фамилия) \\[1.5cm]
        \textbf{Студент} \hfill \underline{\hspace{5cm}} \hfill \und{Д. М. Брюшков} \\
        \hspace{8cm} (Подпись, дата) \hspace{2cm} (И.О.Фамилия)
    \end{flushleft}

\end{titlepage}

\end{document}