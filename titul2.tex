% assignment.tex
\documentclass[main.tex]{subfiles}

\begin{document}

\begin{titlepage}
\centering
\singlespacing % Межстрочный интервал 1.0

{\fontsize{11}{13}\selectfont
\textbf{Министерство науки и высшего образования Российской Федерации}

\textbf{Федеральное государственное автономное образовательное учреждение}

\textbf{высшего образования}

\textbf{«Московский государственный технический университет имени Н.Э. Баумана}

\textbf{(национальный исследовательский университет)»}

\textbf{(МГТУ им. Н.Э. Баумана)}
}

\vspace{0.1cm}

\begin{flushright}
\fontsize{11}{13}\selectfont
УТВЕРЖДАЮ

Заведующий кафедрой \underline{СМ6} \\
(Индекс)

\underline{\hspace{2cm}} \underline{В. М. Кашин} \\
(И.О.Фамилия)

« \underline{\hspace{0.2cm}} » \underline{\hspace{1cm}} 20 \underline{\hspace{0.4cm}} г.
\end{flushright}

\vspace{0.1cm}

\textbf{ЗАДАНИЕ}

\textbf{на выполнение курсовой работы}

\begin{minipage}{\textwidth}
\fontsize{11}{13}\selectfont
по дисциплине \underline{Внутренняя баллистика ракетного и ствольного оружия}.

\vspace{0.1cm}

Студент группы \underline{СМ6-72} \hfill \underline{Ерофеев М.В.} \\
\hfill (Фамилия, имя, отчество)

\vspace{0.1cm}

Тема курсовой работы: \underline{Решение задачи баллистического проектирования} \underline{\hspace{0.5cm}} \\
\underline{\hspace{0.95\textwidth}} \\
\underline{\hspace{0.95\textwidth}}

\vspace{0.1cm}

Направленность КР (учебная, исследовательская, практическая, производственная, др.) \\
\underline{учебная} \hfill \underline{\hspace{0.7\textwidth}}

\vspace{0.1cm}

Источник тематики (кафедра, предприятие, НИР) \\
\underline{кафедра} \hfill \underline{\hspace{0.7\textwidth}}

\vspace{0.1cm}

График выполнения работы: 25\% к \underline{\hspace{0.8cm}} нед., 50\% к \underline{\hspace{0.8cm}} нед., 75\% к \underline{\hspace{0.8cm}} нед., 100\% к \underline{\hspace{0.8cm}} нед.
\end{minipage}

\vspace{0.1cm}

\textbf{Задание \hfill Вариант 24}

\vspace{0.1cm}

\begin{minipage}{\textwidth}
\fontsize{11}{13}\selectfont
\uline{Необходимо провести баллистическое проектирование артиллерийского орудия. То есть определить все возможные значения конструкционных параметров и условий заряжания, обеспечивающих достижение требуемых баллистических характеристик орудия, а затем из этого множества найти оптимальное баллистическое решение по критерию Слухоцкого  $Z_{B1}$. Исходными данными являются: калибр орудия $d$ = 85 мм, масса снаряда $q$ = 5 кг, дульная скорость снаряда $v_{pm}$ = 950 м/с, тип орудия: нарезное. При решении используется квазиодномерная математическая модель. Заданы следующие ограничения: $p_{m}^{\max}$ = 390 МПа, $l_{m}^{\max}$ = 65$d$, $v_{pm - 50}$= 830 м/c $p_{mz + 50}$= 180 МПа. Метательный заряд обязательно должен состоять из нескольких марок порохов.}
\end{minipage}

\vspace{0.1cm}

\begin{flushleft}
\fontsize{11}{13}\selectfont
\textbf{Оформление курсовой работы:}

Расчетно-пояснительная записка на \underline{\hspace{0.6cm}} листах формата А4.

\vspace{0.1cm}

Дата выдачи задания « \underline{\hspace{0.3cm}} » \underline{сентября} 2025 г.
\end{flushleft}

\vfill

\begin{minipage}{\textwidth}
\fontsize{11}{13}\selectfont
\begin{minipage}[t]{0.45\textwidth}
    \raggedright
    \textbf{Руководитель курсовой работы} \\
    \vspace{0.6cm}
    \underline{\hspace{3cm}} \\
    (Подпись, дата)
\end{minipage}
\hfill
\begin{minipage}[t]{0.45\textwidth}
    \raggedleft
    \underline{В. А. Федулов} \\
    (И.О.Фамилия)
\end{minipage}

\vspace{0.2cm}

\begin{minipage}[t]{0.45\textwidth}
    \raggedright
    \textbf{Студент} \\
    \vspace{0.6cm}
    \underline{\hspace{3cm}} \\
    (Подпись, дата)
\end{minipage}
\hfill
\begin{minipage}[t]{0.45\textwidth}
    \raggedleft
    \underline{М.В. Ерофеев} \\
    (И.О.Фамилия)
\end{minipage}

\vspace{0.2cm}

\textbf{Примечание}: Задание оформляется в двух экземплярах: один выдается студенту, второй хранится на кафедре.
\end{minipage}

\end{titlepage}

\end{document}