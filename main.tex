\documentclass[14pt, a4paper]{report} % или article, если не нужен увеличенный базовый шрифт
\usepackage[utf8]{inputenc}
\usepackage[T2A]{fontenc}
\usepackage[russian]{babel}
\usepackage{array}
\usepackage{graphicx}
\usepackage{ulem}
\usepackage{geometry}
\geometry{top=2cm, bottom=2cm, left=2cm, right=2cm}
\usepackage{tabularx}
\usepackage{amsmath}
\usepackage{amssymb}
\usepackage{multirow}
\usepackage{booktabs}
\usepackage{float}
\usepackage{tabularx}
\usepackage{booktabs}
\usepackage{siunitx}
\usepackage{indentfirst}
\usepackage{placeins}
\usepackage{lipsum}
\usepackage{longtable}
\usepackage{csquotes}
\usepackage[style=gost-numeric,sorting=none]{biblatex}
\usepackage{enumitem}
\usepackage{caption}

\captionsetup[table]{position=top, singlelinecheck=false, justification=raggedleft}



\addbibresource{references.bib}
\usepackage{titlesec}
\titleformat{\chapter}[display]
    {\normalfont\huge\bfseries}
    {\chaptertitlename\ \thechapter}
    {20pt}
    {\Huge}
\usepackage{booktabs}
\usepackage{makecell}
\graphicspath{{Image/}}

\geometry{
    a4paper,
    total={170mm,257mm},
    left=20mm,
    top=20mm,
}

\newcommand{\specialcell}[2][c]{%
    \begin{tabular}[#1]{@{}c@{}}#2\end{tabular}%
}

\usepackage{etoolbox}
\patchcmd{\thebibliography}
  {\chapter*{\bibname}}
  {\section*{\centering СПИСОК ЛИТЕРАТУРЫ}}
  {}{}
\patchcmd{\thebibliography}
  {\addcontentsline{toc}{chapter}{\bibname}}
  {\addcontentsline{toc}{section}{СПИСОК ЛИТЕРАТУРЫ}}
  {}{}


\usepackage{setspace}
\onehalfspacing


% Для подчеркивания текста, аналогичного .underline в Word
\newcommand{\und}[1]{\uline{#1}}

\begin{document}

% Титульный лист
\begin{titlepage}
    \centering
    \noindent
    \begin{tabular}{@{}p{0.25\textwidth} p{0.7\textwidth}@{}}
        \raisebox{-\totalheight}{\includegraphics[width=0.8in, height=0.9in]{imgs/BMSTU.jpg}} &
        \textbf{Министерство науки и высшего образования Российской Федерации} \\
        & \\
        & \textbf{Федеральное государственное бюджетное образовательное} \\
        & \textbf{учреждение} \\
        & \\
        & \textbf{высшего образования} \\
        & \\
        & \textbf{«Московский государственный технический университет} \\
        & \\
        & \textbf{имени Н.Э. Баумана} \\
        & \\
        & \textbf{(национальный исследовательский университет)»} \\
        & \\
        & \textbf{(МГТУ им. Н.Э. Баумана)} \\
    \end{tabular}

    \vspace{1cm}
    ФАКУЛЬТЕТ \und{«СПЕЦИАЛЬНОЕ МАШИНОСТРОЕНИЕ»}

    \vspace{0.5cm}
    КАФЕДРА \und{«РАКЕТНЫЕ И ИМПУЛЬСНЫЕ СИСТЕМЫ» (СМ-6)}

    \vspace{2cm}
    \textbf{РАСЧЕТНО-ПОЯСНИТЕЛЬНАЯ ЗАПИСКА}

    \vspace{1cm}
    \textbf{\textit{К КУРСОВОЙ РАБОТЕ}}

    \vspace{1cm}
    \textbf{\textit{НА ТЕМУ:}}

    \vspace{0.5cm}
    \textbf{\textit{\und{«Баллистическое проектирование артиллерийских орудий»}}}

    \vspace{1.5cm}
    \textbf{Вариант 24}

    \vspace{2cm}
    \begin{flushleft}
        Студент \und{СМ6-72} \hfill \underline{\hspace{5cm}} \hfill \und{М.В. Ерофеев} \\
        \hspace{4.8cm} (Группа) \hspace{2.3cm} (Подпись, дата) \hspace{1.8cm} (И.О.Фамилия) \\[1cm]
        Руководитель курсовой работы \hfill \underline{\hspace{5cm}} \hfill \und{В.А. Федулов} \\
        \hspace{7.5cm} (Подпись, дата) \hspace{1.8cm} (И.О.Фамилия)
    \end{flushleft}

    \vfill
    \centering
    \textit{2025 г.}

    \newpage % Начало второй страницы (Задание)

    \centering
    \textbf{Министерство науки и высшего образования Российской Федерации}

    \textbf{Федеральное государственное бюджетное образовательное учреждение}

    \textbf{высшего образования}

    \textbf{«Московский государственный технический университет имени Н.Э. Баумана}

    \textbf{(национальный исследовательский университет)»}

    \textbf{(МГТУ им. Н.Э. Баумана)}

    \vspace{1cm}
    УТВЕРЖДАЮ

    \vspace{0.5cm}
    Заведующий кафедрой \und{СМ6} \\
    \hspace{1.5cm} (Индекс) \\[0.5cm]
    \underline{\hspace{5cm}} \und{В.М.Кашин} \\
    \hspace{0.5cm} (И.О.Фамилия) \\[0.5cm]
    « \underline{\hspace{1cm}} » \underline{\hspace{3cm}} 20 \underline{\hspace{1cm}} г.

    \vspace{2cm}
    \textbf{ЗАДАНИЕ}

    \textbf{на выполнение курсовой работы}

    \vspace{1cm}
    \begin{flushleft}
        по дисциплине \und{Газовая динамика} \hfill \underline{\hspace{7cm}} \\[0.5cm]
        Студент группы \und{СМ6-72} \\[0.5cm]
        \underline{\hspace{12cm}} \und{Ерофеев Максим Викторович} \\
        \hspace{5cm} (Фамилия, имя, отчество) \\[0.5cm]
        Тема курсовой работы \und{Баллистическое проектирование артиллерийских орудий} \\
        \und{на сжатом газе} \\[0.5cm]
        Направленность КР (учебная, исследовательская, практическая, производственная, др.) \\
        \und{Учебная} \hfill \underline{\hspace{7cm}} \\[0.5cm]
        Источник тематики (кафедра, предприятие, НИР) \und{Кафедра} \hfill \underline{\hspace{5cm}} \\[0.5cm]
        График выполнения работы: 25\% к \und{4} нед., 50\% к \und{8} нед., 75\% к \und{12} нед., 100\% к \und{14} нед. \\[1cm]
        \textbf{\textit{Задание}} \und{Найти оптимальные параметры ствола и условий заряжания при решении обратной задачи внутренней баллистики. При этом $d = 85$ мм, $q =$ 5 кг, $ v_{\text{pm}} = 950 $ м/с, тип орудия – НР, тип математической модели – квазиодномерная (КМ), критерий оптимальности – $Z_{\text{B1}}$, $p^{\text{max}}_{\text{m}}$ = 390 МПа, $l^{\text{max}}_{\text{m}}$ = 65 ед.d, $v_{\text{pm-50}}$ = 850 м/с, $p_{\text{mz+5}}0 = $ 180 МПа.} \\[1cm]
        \textbf{\textit{Оформление курсовой работы:}} \\[0.5cm]
        Расчетно-пояснительная записка на \underline{\hspace{2cm}} листах формата А4. \\[0.5cm]
        \underline{\hspace{15cm}} \\
        \underline{\hspace{15cm}} \\
        \underline{\hspace{15cm}} \\
        \underline{\hspace{15cm}} \\[1.5cm]
        Дата выдачи задания «\und{13}» \und{сентября} \und{2024} г. \\[2cm]
        \textbf{Руководитель курсовой работы} \hfill \underline{\hspace{5cm}} \hfill \und{В.А. Федулов} \\
        \hspace{8cm} (Подпись, дата) \hspace{2cm} (И.О.Фамилия) \\[1.5cm]
        \textbf{Студент} \hfill \underline{\hspace{5cm}} \hfill \und{М.В. Ерофеев} \\
        \hspace{8cm} (Подпись, дата) \hspace{2cm} (И.О.Фамилия)
    \end{flushleft}

\end{titlepage}


\tableofcontents

\newpage
\section*{Введение}
Данная курсовая работа посвящена нахождению оптимальных параметров артиллерийского орудия и условий заряжания путем решения обратной задачи внутреннй баллистики.
Ключевым критерием оптимальности решения является критерий качества баллистического решения $Z_{\text{B1}}$. 

Решение должно удовлетворять следующим требованиям: 
\begin{itemize}
    \item  $p_{\text{m}}^{\text{max}}$ $\leq$ 390 МПа
    \item $l_{\text{m}}^{\text{max}}$ $\leq$ 65 ед.d
\end{itemize}

Также на решение наложены следующие ограничения: 
\begin{itemize}
    \item  $v_{\text{pm-50}}$ = 830 $\text{м/c}$
    \item $p_{\text{mz+50}}$ = 180 МПа
\end{itemize}

Условие задания: 
\begin{itemize}
    \item $d$ = 85 мм
    \item $q$ = 5 кг
    \item $v_{\text{pm}}$ = 950 $\text{м/c}$
    \item Тип орудения -- нарезное (НР)
    \item Тип мат. модели -- квазиодномерная (КМ)
\end{itemize}

Данная задача будет решаться с использованием мат. аппарата квазиодномерной модели внутренней.
баллистики. Также использованы методы оптимизации для нахождения оптимального решения обратной задачи с учетом критериев и ограничений.

Вычисления проводились с помощью языка программирования Python с использованием библиотеки PyBallistics [1], визуализация данных осуществлялась 
через библиотеку Matplotlib [2].

\newpage
\chapter{Прямая задача внутренней баллистики}
\section{Описание математической модели}

Выстрел предствавляет собой довольно сложный быстропротекающий физико-химический процесс. Его физическая сущность состоит в том, что при сгорании порохового заряда образуются газообразные продукты сгорания под большим давлением, 
под действием которого снаряд выталкивается из канала ствола с огромной скоростью. Прямая задача состоит в том, чтобы описать движение снаряда массой $q$ по каналу ствола диаметра $d$ под действием давления продуктов сгорания заряда пороха массой $\omega$, находящимся в объеме $W_0$. Схема процесса вместе с качественными распределениями давления и скорости представлена на рисунке 1.
Для упрощения вводится ствол с камерой приведенной длины $l_0$, имеющей тот же обеъем $W_0$, но с диаметром, равным калибру ствола $d$. 

\begin{figure}[h!]
\centering
\includegraphics[width=0.3\textheight]{imgs/1.jpg}
\caption{Схема процесса выстрела}
\end{figure}

Наиболее соверменным и точным описанием процесса выстрела является газодинамический подход, 
по размерности в нашем случае модель является одномерной (квазиодномерной). Эта модель выстрела содержит некоторые допущения: 
\begin{itemize}
    \item Гипотеза односкоростной газопороховой смеси (ОГПС)
    \item Геометрический закон горения пороха
\end{itemize}


    В пространстве между дном ствола и дном снаряда (заснарядный объём) в процессе движения снаряда 
по каналу ствола находятся газообразные продукты сгорания пороха и конденсированные частицы несгоревшего пороха.
Для упрощения принимается, что пороховые газы и конденсированные элементы представляют собой гомогенную смесь, которая движется 
с общей скоростью. Такое допущение называется гипотезой односкоростной газопороховой смеси (ОГПС). Уравнение состояния ОГПС представляется в виде: 

\begin{equation}
\varepsilon = \frac{p}{k - 1} \left( \frac{1}{\rho} - \frac{1 - \psi}{\delta} - b \psi \right) + (1 - \psi) \frac{f}{k - 1},
\label{eq:epsilon}
\end{equation}

где $\varepsilon$ -- удельная внутрення энергия ОГПС, $\rho$ -- плотность ОГПС, $b$ -- коволюм порохового газа (эффективный собственный объём молекул), $k$ -- показатель адиабаты, 
$\psi$ = $\omega_b$ / $\omega$ -- отношение массы сгоревшего элемента у его исходной массе, $\omega_b$ -- масса сгоревшего пороха, $\omega$ -- исходная масса пороха. $\delta$ -- плотность пороха. 

Геометрический закон горения пороха выражается в виде формулы (1): 

\begin{equation}
\frac{dz}{dt} = \frac{p^\nu}{I_e}, 
\label{eq:1}
\end{equation}

где  $p$ -- давление газа, $\nu$ -- показатель степени
в законе горения. В артиллерии, как правило, $\nu$ = 1. $z$ = $e$ /$e_1$ -- безразмерная толщина сгоревшего свода порохового элемента. В свою очередь
$e$ -- координата текущего положения поверхности горения, а $e_1$ -- полная толщина горящего свода порохового элемента. $I_e$ -- полный импульс давления пороховых газов:
\[
I_e = \int_0^{t_e} p^v  dt = \frac{e_1}{u_1},
\]

где $u_1$ -- скорость горения при единичном горении, определяемая экспериментальным путем.

Далее рассмотрим систему уравнений для газодинамической задачи в приближении ОГПС. ОГПС в данном случае представляет собой «псевдогаз», её движение в заснарядном
объёме описывается стандартными уравнениями сохранения массы, импульса и энергии в лагранжевых координатах: 

\begin{equation}
\frac{\partial v}{\partial m} = \frac{d}{dt}\left(\frac{1}{\rho S}\right)
\end{equation}

\begin{equation}
\frac{dv}{dt} + S\frac{\partial p}{\partial m} = 0,
\end{equation}

\begin{equation}
\frac{d\epsilon}{dt} + p\frac{\partial}{\partial m}(vS) = 0.
\end{equation}

Здесь $m$ -- массовая лагранжева координата.









\newpage
\begin{thebibliography}{9}
\bibitem{} \textbf{PyBallistics}
\bibitem{} \textbf{Matplotlib} 


\end{thebibliography}

\end{document}